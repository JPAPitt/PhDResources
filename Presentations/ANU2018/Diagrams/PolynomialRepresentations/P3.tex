\documentclass[pdf]{standalone}
\usepackage{pgfplots} 
%\usepgfplotslibrary{external} 
%\tikzexternalize 
\usetikzlibrary{arrows, decorations.markings}
\usepackage{tikz} 
\usepackage{amsmath} 
\usetikzlibrary{calc} 
\pgfplotsset{compat = newest}
\begin{document} 
	\begin{tikzpicture}
	\makeatletter
	\tikzset{
		nomorepostaction/.code=\makeatletter\let\tikz@postactions\pgfutil@empty, % From https://tex.stackexchange.com/questions/3184/applying-a-postaction-to-every-path-in-tikz/5354#5354
		my axis/.style={
			postaction={
				decoration={
					markings,
					mark=at position 1 with {
						\arrow[ultra thick]{latex}
					}
				},
				decorate,
				nomorepostaction
			},
			thin,
			-, % switch off other arrow tips
			every path/.append style=my axis % this is necessary so it works both with "axis lines=left" and "axis lines=middle"
		}
	}
	\makeatother
	
	\begin{axis}[ 
	axis lines = left, 
	axis line style={my axis},
	width=15cm,
	height = 7.5cm,
	xtick={1,3,5,7,9},  
	ytick = {\empty},
	xticklabels ={$x_0$,$x_1$,$x_2$,$x_3$,$x_4$},
	yticklabels ={},
	xmin=0, 
	xmax=10, 
	ymin =-0.25, 
	ymax = 1,
	xlabel=$x$, 
	ylabel=$b(x)$ ]
		
	\draw[dashed,thick] (2, -0.25  ) -- (2,1);
	\draw[dashed,thick] (4, -0.25  ) -- (4,1);
	\draw[dashed,thick] (6, -0.25  ) -- (6,1);
	\draw[dashed,thick] (8, -0.25  ) -- (8,1);
		
	\addplot[domain=0:2,samples=25, smooth, thick,green!60!black] gnuplot[id=poly1]{0.2*(x)*(x-1)*(x-1)};
	\addplot[domain=2:4,samples=25, smooth, thick,green!60!black] gnuplot[id=poly2]{0.1*(x-2)*(x-5)*(x-5) + 0.4 };
	\addplot[domain=4:6,samples=25, smooth, thick,green!60!black] gnuplot[id=poly3]{-0.05*(x-4)*(x-4)*(x-4) + 0.6  };
	\addplot[domain=6:8,samples=25, smooth, thick,green!60!black] gnuplot[id=poly4]{0.1*(6-x)*(7.5 - x)*(x-5.5) + 0.2 };		
	\addplot[domain=8:10,samples=25, smooth, thick,green!60!black] gnuplot[id=poly5]{0.1*(x-8)*(x-9)*(x-9) + 0.45};
	
	\node[label={90:{\Large$b_0(x)$}}] at (axis cs:1,0.8) {};	
	\node[label={90:{\Large$b_1(x)$}}] at (axis cs:3,0.8) {};
	\node[label={90:{\Large$b_2(x)$}}] at (axis cs:5,0.8) {};
	\node[label={90:{\Large$b_3(x)$}}] at (axis cs:7,0.8) {};
	\node[label={90:{\Large$b_4(x)$}}] at (axis cs:9,0.8) {};					
		
	\end{axis} 
	\end{tikzpicture} 
\end{document}